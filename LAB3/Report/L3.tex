\documentclass[a4paper,11pt]{article}

\usepackage{wrapfig}
\usepackage{amsmath}
\usepackage{bm}
\usepackage{pgfplots}
\usepackage{enumerate}
\usepackage{xcolor}
\usepackage[most]{tcolorbox}
%Russian-specific packages
\usepackage[T2A]{fontenc}
\usepackage[utf8]{inputenc}
\usepackage[russian, english]{babel}
%--------------------------------------
\usepackage{color} %% это для отображения цвета в коде
\usepackage{listings} %% собственно, это и есть пакет listings
\usepackage{caption}

\DeclareCaptionFont{white}{\color{white}} %% это сделает текст заголовка белым
%код ниже нарисует серую рамочку вокруг заголовка кода.
\DeclareCaptionFormat{listing}{\colorbox{gray}{\parbox{\textwidth}{#1#2#3}}}
\captionsetup[lstlisting]{format=listing,labelfont=white,textfont=white}
%--------------------------------------
\definecolor{lemonchiffon}{rgb}{1.0, 0.98, 0.8}
\newtcolorbox{mainblock}{colback=lemonchiffon,grow to right by=-10mm,grow to left by=-10mm,
boxrule=0pt,boxsep=0pt,breakable} % настройки области с изменённым фоном
%--------------------------------------
\definecolor{block-gray}{gray}{0.95} % уровень прозрачности (1 - максимум)
\newtcolorbox{importantblock}{colback=block-gray,grow to right by=-10mm,grow to left by=-10mm,
boxrule=0pt,boxsep=0pt,breakable} % настройки области с изменённым фоном
%--------------------------------------
\makeatletter
\newcommand{\settag}[1]{
  \tag*{(#1)\qquad}
  \edef\@currentlabel{\theequation#1}}
\makeatother

%---------code stuff-------------------
\lstset{
language=C++,                 % выбор языка для подсветки (здесь это С)
basicstyle=\footnotesize\sffamily, % размер и начертание шрифта для подсветки кода
numbers=left,               % где поставить нумерацию строк (слева\справа)
numberstyle=\tiny,           % размер шрифта для номеров строк
stepnumber=1,                   % размер шага между двумя номерами строк
numbersep=5pt,                % как далеко отстоят номера строк от подсвечиваемого кода
backgroundcolor=\color{white}, % цвет фона подсветки - используем \usepackage{color}
showspaces=false,            % показывать или нет пробелы специальными отступами
showstringspaces=false,      % показывать или нет пробелы в строках
showtabs=false,             % показывать или нет табуляцию в строках
frame=single,              % рисовать рамку вокруг кода
tabsize=2,                 % размер табуляции по умолчанию равен 2 пробелам
captionpos=t,              % позиция заголовка вверху [t] или внизу [b]
breaklines=true,           % автоматически переносить строки (да\нет)
breakatwhitespace=false, % переносить строки только если есть пробел
escapeinside={\%*}{*)}   % если нужно добавить комментарии в коде
}
%--------------------------------------

%---------TITLE PAGE---------
\title{Вычислительная математика \\ Лабораторная работа 3}
\author{Ярослав Пылаев, гр. 3530904/80001}
\date{March 31, 2020}

\begin{document}
\maketitle
\newpage

\section{Задание}
\noindent \textbf{Вариант 14.} Решить систему дифференциальных уравнений:
  \begin{gather*}
      \frac{dx_1}{dt} = -73x_1 - 210x_2 + ln(1 + t^2), \hspace{5mm}
      \frac{dx_2}{dt} = x_1 + e^{-t} + t^2 + 1, \\
    x_1(0) = -3, \hspace{3mm} x_2(0) = 1, \hspace{3mm} t \in [0; 1],
  \end{gather*}
  следующими способами с одним и тем же шагом печати $h_{print} = 0.05$:
  \begin{enumerate}[I.]
    \item по программе $\verb|RKF45|$ с $EPS = 0.0001$;
    \item методом Адамса $2$-й степени точности: \[z_{n+1} = z_n + h(3f_n - f_{n-1}) / 2\]
          с двумя постоянными шагами интегрирования:
          \begin{itemize}
            \item $h_{int} = 0.025$,
            \item любой другой, позволяющий получить качественно верное решение.
          \end{itemize}
  \end{enumerate}

  \noindent Сравнить результаты. Дополнительные начальные условия метода Адамса получить с помощью $\verb|RKF45|$.

\subsection{Цель работы}
\noindent Решить систему дифференциальных уравнений сначала по программе $\verb|RKF45|$, а затем методом
      Адамса $2$-й степени точности (с двумя различными шагами интегрирования). Сравнить полученные результаты.

\subsection{Задачи}
\begin{enumerate}
  \item Используя программу $\verb|RKF45|$, решить исходную систему дифференциальных уравнений.
  \item Вычислить начальные условия для метода Адамса $2$-й степени точности.
  \item Решить СДУ методом Адамса с шагами интегрирования
    \begin{itemize}
      \item $0.025$,
      \item выбранный с учетом расчета критического шага метода.
    \end{itemize}
    \item Сравнить полученные результаты.
\end{enumerate}
\newpage

\section{Реализация}
\subsection{Инициализация}
\begin{lstlisting}[label=start, caption=Starting values]
  int main()
  {
    const double startValue1 = -3.0;
    const double startValue2 = 1.0;
    const int N = 2;
    int fail = 0;
    rkfinit(N, &fail); //RKFINIT()
    if (fail) {
      return 1;
    }
    //---other arguments for RKF45()
    double arrValues[N] = {startValue1, startValue2};
    double arrDxDt[N] = {0.0};
    double t = 0.0;
    double tOut = 0.0;
    double relerr = 1.0e-4;
    double abserr = 1.0e-4;
    double h = 0.0;
    int nfe = 0;
    const int maxnfe = 5000; //take value from example
    int flag = 1;
    //---
    ......................................................
\end{lstlisting}

\subsection{RKF45}
\begin{lstlisting}[label=rkf45, caption=RKF45]
    ......................................................
    //compute number of iterations for the loop
    const double HPRINT = 5.0e-2;
    const double t1 = 0.0;
    const double t2 = 1.0;
    int nIter = static_cast<int>(abs((t2-t1)) / HPRINT);
    for (int i = 0; i < nIter; ++i) {
      tOut += HPRINT;
      rkf45(fun, N, arrValues, arrDxDt, &t, tOut, &relerr,
            abserr, &h, &nfe, maxnfe, &flag);
    }
    ......................................................
\end{lstlisting}
\newpage
\subsubsection{Результаты}
  $h_{print} = 0.05$ \\
  \begin{center}
    \begin{tabular}{ | c | c | c | }
      \hline
      %\footnotesize
      $\textbf{t}$ & \bm{$x_1$} & \bm{$x_2$} \\ \hline
      $0.05$ & $\text{-}2.78428$ & $0.95533$ \\ \hline
      $0.1$ & $\text{-}2.66501$ & $0.91586$ \\ \hline
      $0.15$ & $\text{-}2.55931$ & $0.88022$ \\ \hline
      $0.2$ & $\text{-}2.46418$ & $0.84829$ \\ \hline
      $0.25$ & $\text{-}2.379$ & $0.81963$ \\ \hline
      $0.3$ & $\text{-}2.30346$ & $0.79438$ \\ \hline
      $0.35$ & $\text{-}2.23712$ & $0.77232$ \\ \hline
      $0.4$ & $\text{-}2.1797$ & $0.75335$ \\ \hline
      $0.45$ & $\text{-}2.13087$ & $0.73736$ \\ \hline
      $0.5$ & $\text{-}2.09039$ & $0.72425$ \\ \hline
      $0.55$ & $\text{-}2.05799$ & $0.71395$ \\ \hline
      $0.6$ & $\text{-}2.03349$ & $0.70637$ \\ \hline
      $0.65$ & $\text{-}2.01665$ & $0.70146$ \\ \hline
      $0.7$ & $\text{-}2.00733$ & $0.69914$ \\ \hline
      $0.75$ & $\text{-}2.00533$ & $0.69937$ \\ \hline
      $0.8$ & $\text{-}2.01052$ & $0.70208$ \\ \hline
      $0.85$ & $\text{-}2.02276$ & $0.70723$ \\ \hline
      $0.9$ & $\text{-}2.04192$ & $0.71478$ \\ \hline
      $0.95$ & $\text{-}2.06788$ & $0.72469$ \\ \hline
      $1$ & $\text{-}2.10053$ & $0.73691$ \\ \hline
    \end{tabular} \\
  \end{center}

\subsection{Метод Адамса 2-й степени точности}
\begin{equation*}
  z_{n+1} = z_n + h(3f_n - f_{n-1}) / 2
\end{equation*}
\begin{lstlisting}[label=adams, caption=Adams (2) method (Linear multistep method)]
    ......................................................
    double fN[N] = {0.0}; //f_n
    double fNm1[N] = {0.0}; //f_{n-1}

  //---compute all starting values
    arrValues[0] = startValue1;
    arrValues[1] = startValue2;
    fun(N, t1, arrValues, fN); //compute f_n
    const double HINT = 2.5e-3; //change this value by int step

    //we know x_1(0) and x_2(0) but also need x_1(0 - HINT) and
          x_2(0 - HINT) as starting valuses for Adams method
    tOut = t1 - HINT; //it's less than 0
    t = t1; //it's 0
    //tOut <-- t (not t --> tOut)
    rkf45(fun, N, arrValues, arrDxDt, &t, tOut, &relerr,
          abserr, &h, &nfe, maxnfe, &flag);
    fun(N, t, arrValues, fNm1); //compute f_{n-1}
  //---

    nIter = static_cast<int>(abs((t2-t1)) / HINT);
    t = t1;
    for (int i = 0; i < nIter; ++i) {
      for (int j = 0; j < N; ++j) {
        arrValues[j] = arrValues[j] + 0.5 * HINT * (3 * fN[j] - fNm1[j]); //formula for the method
        fNm1[j] = fN[j];
      }
      t += HINT;
      fun(N, t, arrValues, fN);
    }
    rkfend();

    return 0;
  }
\end{lstlisting}

\subsubsection{Расчет критического шага}
\begin{gather*}
  \frac{dx_1}{dt} = -73x_1 - 210x_2 + ln(1 + t^2), \hspace{5mm}
  \frac{dx_2}{dt} = x_1 + e^{-t} + t^2 + 1.
\end{gather*}
\begin{align*}
  &\begin{pmatrix}
    $\text{--}73$ & $\text{--}210$ \\
    $1$ & 0
   \end{pmatrix}, \hspace{5mm}
   \lambda^2 + 73\lambda + 210 = 0, \hspace{5mm}
   \begin{cases}
     \lambda_1 = -3, \\
     \lambda_2 = -70.
   \end{cases} \\
   &\text{При условии, что } h_{int}|\lambda_k| < 1, \bm{h_{int} < 0.01428571...} \\
\end{align*}

\subsubsection{Результаты}
\begin{enumerate}[I.]
  \item $h_{int} = 0.025$
  \begin{center}
    \begin{tabular}{ | c | c | c | }
      \hline
      %\footnotesize
      $\textbf{t}$ & \bm{$x_1$} & \bm{$x_2$} \\ \hline
      $0.025$ & $\text{-}3.48602$ & $1.00494$ \\ \hline
      $0.05$ & $\text{-}1.96944$ & $0.96082$ \\ \hline
      $0.075$ & $\text{-}4.68745$ & $0.97910$ \\ \hline
      $0.1$ & $1.15922$ & $0.87602$ \\ \hline
      $0.125$ & $\text{-}10.6197$ & $1.02572$ \\ \hline
      $0.15$ & $13.7319$ & $0.66024$ \\ \hline
      $0.175$ & $\text{-}36.0558$ & $1.35483$ \\ \hline
      $0.2$ & $66.2421$ & $\text{-}0.12233$ \\ \hline
      $0.225$ & $\text{-}143.476$ & $2.85877$ \\ \hline
      $0.25$ & $286.902$ & $\text{-}3.30348$ \\ \hline
      $\cdots$ & $\cdots$ & $\cdots$ \\ \hline
    \end{tabular} \\
  \end{center}
  \noindent Очевидно, что при заданном шаге интегрирования получается качественно неверное решение.

  \item Пусть $h_{int} = 0.0025$, тогда
  \begin{align*}
    &\begin{tabular}{ | c | c | c | }
      \hline
      %\footnotesize
      $\textbf{t}$ & \bm{$x_1$} & \bm{$x_2$} \\ \hline
      $0.0025$ & $-2.99958$ & $0.999666$ \\ \hline
      $0.005$ & $-2.97693$ & $0.997158$ \\ \hline
      $0.0075$ & $-2.95856$ & $0.994729$ \\ \hline
      $0.01$ & $-2.94189$ & $0.992334$ \\ \hline
      $0.0125$ & $-2.92686$ & $0.989972$ \\ \hline
      $0.015$ & $-2.9132$ & $0.98764$ \\ \hline
      $0.0175$ & $-2.90068$ & $0.985334$ \\ \hline
      $0.02$ & $-2.88915$ & $0.983053$ \\ \hline
      $0.0225$ & $-2.87843$ & $0.980793$ \\ \hline
      $0.025$ & $-2.86842$ & $0.978553$ \\ \hline
      $0.0275$ & $-2.859$ & $0.976331$ \\ \hline
      $0.03$ & $-2.85008$ & $0.974127$ \\ \hline
      $0.0325$ & $-2.84159$ & $0.971938$ \\ \hline
      $0.035$ & $-2.83346$ & $0.969765$ \\ \hline
      $0.0375$ & $-2.82565$ & $0.967606$ \\ \hline
      $0.04$ & $-2.8181$ & $0.96546$ \\ \hline
      $0.0425$ & $-2.81079$ & $0.963327$ \\ \hline
      $0.045$ & $-2.80367$ & $0.961207$ \\ \hline
      $0.0475$ & $-2.79672$ & $0.959099$ \\ \hline
      $0.05$ & $-2.78992$ & $0.957003$ \\ \hline
      $0.0525$ & $-2.78325$ & $0.954919$ \\ \hline
      $0.055$ & $-2.77669$ & $0.952845$ \\ \hline
      $0.0575$ & $-2.77024$ & $0.950783$ \\ \hline
      $0.06$ & $-2.76387$ & $0.948731$ \\ \hline
      $0.0625$ & $-2.75758$ & $0.94669$ \\ \hline
      $0.065$ & $-2.75136$ & $0.94466$ \\ \hline
      $0.0675$ & $-2.74521$ & $0.94264$ \\ \hline
      $0.07$ & $-2.73911$ & $0.94063$ \\ \hline
      $0.0725$ & $-2.73307$ & $0.938631$ \\ \hline
      $0.075$ & $-2.72708$ & $0.936642$ \\ \hline
      $\cdots$ & $\cdots$ & $\cdots$ \\ \hline
    \end{tabular}
    &\begin{tabular}{ | c | c | c | }
      \hline
      %\footnotesize
      $\textbf{t}$ & \bm{$x_1$} & \bm{$x_2$} \\ \hline
      $0.05$ & $-2.78992$ & $0.957003$ \\ \hline
      $0.1$ & $-2.66935$ & $0.917293$ \\ \hline
      $0.15$ & $-2.56302$ & $0.881451$ \\ \hline
      $0.2$ & $-2.46734$ & $0.849254$ \\ \hline
      $0.25$ & $-2.38174$ & $0.82054$ \\ \hline
      $0.3$ & $-2.3058$ & $0.795166$ \\ \hline
      $0.35$ & $-2.23914$ & $0.773002$ \\ \hline
      $0.4$ & $-2.18143$ & $0.753934$ \\ \hline
      $0.45$ & $-2.13236$ & $0.737858$ \\ \hline
      $0.5$ & $-2.09167$ & $0.724681$ \\ \hline
      $0.55$ & $-2.0591$ & $0.714318$ \\ \hline
      $0.6$ & $-2.03443$ & $0.706692$ \\ \hline
      $0.65$ & $-2.01747$ & $0.701734$ \\ \hline
      $0.7$ & $-2.00802$ & $0.69938$ \\ \hline
      $0.75$ & $-2.00593$ & $0.699572$ \\ \hline
      $0.8$ & $-2.01104$ & $0.702256$ \\ \hline
      $0.85$ & $-2.02321$ & $0.707385$ \\ \hline
      $0.9$ & $-2.0423$ & $0.714913$ \\ \hline
      $0.95$ & $-2.06821$ & $0.724799$ \\ \hline
      $1$ & $-2.10081$ & $0.737004$ \\ \hline
    \end{tabular}
  \end{align*}
  Теперь результаты нас удовлетворяют.
\end{enumerate}
\newpage
\section{Вывод}
В ходе выполнения лабораторной работы мы получили результаты решения системы дифференциальных уравнений, вычисленных, используя программу $\verb|RKF45|$ с шагом печати $0.05$, и по методу Адамса с шагом, удовлетворяющим критическому --- $0.0025 < \frac{1}{70}$. Значения результатов оказались крайне близкими. Также убедились, что при шаге интегрирования больше критического решение является качественно неверным.

\section{Дополнительные процедуры}
\begin{lstlisting}[label=fun, caption=Function]
  int fun(int n, double t, double *arrValues, double *arrDxDt)
  {
    for (int i = 0; i < n; ++i) {
      switch (i)
      {
        //X_1 = arrValues[0], X_2 = arrValues[1]
        case 0 :
          arrDxDt[i] = (-73) * arrValues[0] - 210 * arrValues[1] + log(1 + pow(t, 2));
          break;

        case 1 :
          arrDxDt[i] = arrValues[0] + exp(-t) + pow(t, 2) + 1;
          break;

        default:
          std::cerr << "The number of ODE's is bigger than known count of equations!";
          return 1;
      }
    }
    return 0;
  }
\end{lstlisting}

\end{document}
