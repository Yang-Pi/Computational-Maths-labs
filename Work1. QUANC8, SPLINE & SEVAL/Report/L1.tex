\documentclass[a4paper,11pt]{article}

\usepackage{wrapfig}
\usepackage{amsmath}
\usepackage{pgfplots}
\usepackage{enumerate}
\usepackage{xcolor}
\usepackage[most]{tcolorbox}
%Russian-specific packages
%--------------------------------------
\usepackage[T2A]{fontenc}
\usepackage[utf8]{inputenc}
\usepackage[russian, english]{babel}
%--------------------------------------
\usepackage{color} %% это для отображения цвета в коде
\usepackage{listings} %% собственно, это и есть пакет listings
\usepackage{caption}
\DeclareCaptionFont{white}{\color{white}} %% это сделает текст заголовка белым
%% код ниже нарисует серую рамочку вокруг заголовка кода.
\DeclareCaptionFormat{listing}{\colorbox{gray}{\parbox{\textwidth}{#1#2#3}}}
\captionsetup[lstlisting]{format=listing,labelfont=white,textfont=white}
%--------------------------------------
\definecolor{lemonchiffon}{rgb}{1.0, 0.98, 0.8}
\newtcolorbox{mainblock}{colback=lemonchiffon,grow to right by=-10mm,grow to left by=-10mm,
boxrule=0pt,boxsep=0pt,breakable} % настройки области с изменённым фоном

\definecolor{block-gray}{gray}{0.95} % уровень прозрачности (1 - максимум)
\newtcolorbox{importantblock}{colback=block-gray,grow to right by=-10mm,grow to left by=-10mm,
boxrule=0pt,boxsep=0pt,breakable} % настройки области с изменённым фоном

\makeatletter
\newcommand{\settag}[1]{
  \tag*{(#1)\qquad}
  \edef\@currentlabel{\theequation#1}}
\makeatother

\title{Вычислительная математика \\ Лабораторная работа 1}
\author{Ярослав Пылаев, гр. 3530904/80001}
\date{\today}

\begin{document}
\maketitle

%code stuff---------
\lstset{
language=C++,                 % выбор языка для подсветки (здесь это С)
basicstyle=\footnotesize\sffamily, % размер и начертание шрифта для подсветки кода
numbers=left,               % где поставить нумерацию строк (слева\справа)
numberstyle=\tiny,           % размер шрифта для номеров строк
stepnumber=1,                   % размер шага между двумя номерами строк
numbersep=5pt,                % как далеко отстоят номера строк от подсвечиваемого кода
backgroundcolor=\color{white}, % цвет фона подсветки - используем \usepackage{color}
showspaces=false,            % показывать или нет пробелы специальными отступами
showstringspaces=false,      % показывать или нет пробелы в строках
showtabs=false,             % показывать или нет табуляцию в строках
frame=single,              % рисовать рамку вокруг кода
tabsize=2,                 % размер табуляции по умолчанию равен 2 пробелам
captionpos=t,              % позиция заголовка вверху [t] или внизу [b]
breaklines=true,           % автоматически переносить строки (да\нет)
breakatwhitespace=false, % переносить строки только если есть пробел
escapeinside={\%*}{*)}   % если нужно добавить комментарии в коде
}
%--------------------------------------

\newpage
\section{Задание}
\noindent \textbf{Вариант 14.} Для $1 \le x \le 4$ с $h = 0.375$ вычислить значения
      \[ f(x) = \int_0^{20} \frac{dz}{e^z(z + x)}, \]
      используя для вычисления интеграла программу $\verb|QUANC8|$.
      По полученным точкам построить сплайн-функцию и полином Лагранжа 8-й степени.
      Сравнить значения обеих аппроксимаций в точках \[ x_k = 1.1875 + 0.375k, \hspace{5mm} (k = 0, 1,\dots, 7).\]
\subsection{Цель работы}
\noindent Вычислить значение заданного интеграла и исследовать точность аппроксимирующих функций,
      полученных
\begin{itemize}
  \item сплайн-интерполяцией,
  \item построением интерполяционного полинома Лагранжа.
\end{itemize}
\subsection{Задачи}
\begin{enumerate}
  \item Используя программу $\verb|QUANC8|$ определить значение интеграла.
  \item При помощи программ $\verb|SPLINE|$ и $\verb|SEVAL|$ построить сплайн-функцию.
  \item Построить интерполяционный полином Лагранжа.
  \item Сравнить полученные значения в точках $x_k$,
        дополнительно для определения точности аппроксимаций для этих же точек применив программу $\verb|QUANC8|$.
\end{enumerate}

\newpage
\section{Реализация}
\subsection{QUANC8}

\begin{lstlisting}[label=quanc8,caption=QUANC8]
const int X_LIMIT_1 = 1;
const int X_LIMIT_2 = 4;
const double H = 0.375;

const double A = 0;
const double B = 20;

int main()
{
//other INPUT parameters
  const double epsabs = 0.0;
  const double epsrel = 1.0e-10;

//OUTPUT parameters
  double result = 0.0;
  double errest = 0.0;
  int nofun = 0;
  double posn = 0.0;
  int flagQuanc = 0;

  const int n = 1 + static_cast<int>(abs(X_LIMIT_1 -
        - X_LIMIT_2) / H);

  double arrQuancRes[n] = { 0 };

  for (int i = 0; i < n; ++i) {
    quanc8(fun, A, B, epsabs, epsrel, &result,
          &errest, &nofun, &posn, &flagQuanc); //QUANC8
    arrQuancRes[i] = result;
  }
  .........................
\end{lstlisting}

\begin{lstlisting}[label=fun,caption=Подынтегральная функция]
double fun(double z)
{
  double x = X_LIMIT_1;
  static int i = -1;
  if (z == A) {
    x = X_LIMIT_1 + (++i * H); //compute x
  }
  else {
    x = X_LIMIT_1 + (i * H);
  }
  return { 1 / (exp(z) * (z + x)) };
}
\end{lstlisting}

\subsubsection{Результат}
\begin{tabular}{ | c | c | c | c | c | c | c | c | c | c | }
  \hline
  \footnotesize
  $x$ & 1 & 1.375 & 1.75 & 2.125 & 2.5 & 2.875 & 3.25 & 3.625 & 4 \\ \hline
  $f(x)$ & 0.5963 & 0.4774 & 0.3998 & 0.3448 & 0.3035 & 0.2713 & 0.2454 & 0.2241 & 0.2063 \\ \hline
\end{tabular} \\

$\verb|FLAG QUANC8| = 0$

\subsubsection{Комментарии}
\noindent Так как подынтегральная функция зависит от двух переменных,
      то программа $\verb|QUANC8|$ работает относительно переменной $z$,
      а изменение $x$ учитывается в самой функции
      (используется статическая переменная для расчета). \\

\noindent Результаты программы являются разумными, потому что при увеличении переменной $x$,
      находещейся в знаменателе, значение функции уменьшается.

\vspace{5mm}
\subsection{Сплайн-интерполяция}
\begin{lstlisting}[label=spline,caption=SPLINE-SEVAL]
  .........................
  const int k_min = 0;
  const int k_max = 7;
  const int nK = k_max - k_min + 1;

  double arrSplineRes[nK] = { 0 };

  //compute X_k values
  double arrXKPoints[nK];
  for (int i = 0; i < nK; ++i) {
    arrXKPoints[i] = 1.1875 + 0.375 * i;
  }

  double arrB[nK] = { 0 };
  double arrC[nK] = { 0 };
  double arrD[nK] = { 0 };

  int flagSpline = 0;
  int last = 0;

  spline(n, 1, 1, 1, 1, arrXPoints, arrQuancRes,
        arrB, arrC, arrD, &flagSpline); //SPLINE

  for (int i = k_min; i < k_max + 1; ++i) {
    arrSplineRes[i] = seval(n, arrXKPoints[i], arrXPoints,
          arrQuancRes, arrB, arrC, arrD, &last); //SEVAL
  }
  .........................
\end{lstlisting}

\subsubsection{Результат}
\begin{tabular}{ | c | c | c | c | c | c | c | c | c | }
  \hline
  \footnotesize
  $x_k$ & 1.1875 & 1.5625 & 1.9375 & 2.3125 & 2.6875 & 3.0625 & 3.4375 & 3.8125 \\ \hline
  $f(x)$ & 0.5963 & 0.4193 & 0.3744 & 0.3213 & 0.2879 & 0.2531 & 0.2509 & 0.2268 \\ \hline
\end{tabular} \\

$\verb|FLAG SPLINE| = 0$

\vspace{5mm}
\subsection{Интерполяционный полином Лагранжа}
\begin{lstlisting}[label=lagrange,caption=Построение полинома Лагранжа]
  .........................
  double arrLagrRes[nK] = { 0 };

  for (int i = k_min; i < k_max + 1; ++i) {
    arrLagrRes[i] = lagrange(n, arrXPoints, arrQuancRes,
          nK - 1, arrXKPoints[i]); //LAGRANGE
  }
  .........................
\end{lstlisting}

\begin{lstlisting}[label=lagrange,caption=lagrange.cpp]
double lagrange(int n, double arrPoints[],
      double arrFunctions[], int m, double x)
{
  if (n < m) {
    throw std::invalid_argument("There are not enough points");
  }
  if (n <= 0) {
    throw std::invalid_argument("Tables are empty");
  }
  //find nodes satisfying the condition of interpolation
  int h = 0;
  for (int i = 0; i < n; i++) {
    if ((arrPoints[i] > x) && (i >= m)) {
        h = i + 1 - m;
        break;
    }
  }
  //compute
  double q = 0.0;
  for (int i = h; i < h + m; i++) {
    double p = 1.0;
    for (int j = h; j < h + m; j++) {
      if (j != i) {
        p *= x - arrPoints[j];
        p /= arrPoints[i] - arrPoints[j];
      }
    }
    p *= arrFunctions[i];
    q += p;
  }
  return q;
}
\end{lstlisting}

\subsubsection{Результат}
\begin{tabular}{ | c | c | c | c | c | c | c | c | c | }
  \hline
  \footnotesize
  $x_k$ & 1.1875 & 1.5625 & 1.9375 & 2.3125 & 2.6875 & 3.0625 & 3.4375 & 3.8125 \\ \hline
  $f(x)$ & 0.5293 & 0.4350 & 0.3702 & 0.3228 & 0.2864 & 0.2577 & 0.2342 & 0.2148 \\ \hline
\end{tabular} \\
\subsubsection{Комментарии}
\noindent Для построения интерполяционного полинома Лагранжа была использована программа $\verb|lagrange.cpp|$,
      написанная еще в 3-м семестре в рамках курса по вычислительной математике, которая
      самостоятельно способна       выбирать наиболее подходящие узлы интерполирования.
      Даже несмотря на то, что в решаемой задаче выбор узлов не предоставляется (9 точек, требуемая степень - 8),
      упомянуть о свойстве программы имеет смысл.

\vspace{5mm}
\subsection{Дополнительные расчеты}
\begin{lstlisting}[label=quanc8 extra,caption=QUANC8 по точкам $x_k$]
  .........................
  for (int i = 0; i < nK; ++i) {
    quanc8(checkFun, A, B, epsabs, epsrel, &result,
          &errest, &nofun, &posn, &flagQuanc);
    arrQuancRes[i] = result;
  }

  return 0;
}
\end{lstlisting}

\begin{lstlisting}[label=fun extra,caption=Функция с расчетом $x_k$]
double checkFun(double z)
{
  double x = 1.1875;
  static int i = -1;
  if (z == A) {
    x = 1.1875 + 0.375 * (++i);
  }
  else {
    x = 1.1875 + 0.375 * i;
  }

  return { 1 / (exp(z) * (z + x)) };
}
\end{lstlisting}

\subsubsection{Результат}
\begin{tabular}{ | c | c | c | c | c | c | c | c | c | }
  \hline
  \footnotesize
  $x_k$ & 1.1875 & 1.5625 & 1.9375 & 2.3125 & 2.6875 & 3.0625 & 3.4375 & 3.8125 \\ \hline
  $f(x)$ & 0.5298 & 0.4350 & 0.3702 & 0.3228 & 0.2864 & 0.2577 & 0.2343 & 0.2148 \\ \hline
\end{tabular} \\

\subsection{Сравнение результатов}
\begin{tabular}{ | c | c | c |}
  \hline
  \footnotesize
  $x_k$ & $\triangle f(x)_S$ & $\triangle f(x)_L$ \\ \hline
  1.1875 & -0.0665 & 0.0004 \\ \hline
  1.5625 & 0.0156 & $-2.8\cdot10^{-5}$ \\ \hline
  1.9375 & -0.0042 & $6.4\cdot10^{-6}$ \\ \hline
  2.3125 & 0.0014 & $-3.0\cdot10^{-6}$ \\ \hline
  2.6875 & -0.0014 & $2.6\cdot10^{-6}$ \\ \hline
  3.0625 & 0.0045 & $-4.2\cdot10^{-6}$ \\ \hline
  3.4375 & -0.0166 & $1.4\cdot10^{-5}$ \\ \hline
  3.8125 & -0.0119 & $-4.8\cdot10^{-6}$ \\ \hline
\end{tabular} \\

\noindent $\triangle f(x)_{S|L} = f(x)_{Q8} - f(x)_{S|L}$

\vspace{5mm}
\section{Вывод}
\noindent Значение заданного интеграла было успешно вычисленно с помощью программы $\verb|QUANC8|$,
      а в результате сплайн-интерполяции и построения полинома Лагранжа 8-й степени были получены аппроксимации
      исходной функции, что позволяет исследовать ее поведение. Проанализировав таблицу п. 2.5 "Сравнение результатов",
      можно сделать вывод о точности методов, заключачающийся в том, что значения аппроксимации, полученной построением
      полинома Лагранжа, более точные, чем значения, полученные сплайн-интерполяцией.

\end{document}
