\documentclass[a4paper,11pt]{article}

\usepackage{wrapfig}
\usepackage{amsmath}
\usepackage{pgfplots}
\usepackage{enumerate}
\usepackage{xcolor}
\usepackage[most]{tcolorbox}
%Russian-specific packages
\usepackage[T2A]{fontenc}
\usepackage[utf8]{inputenc}
\usepackage[russian, english]{babel}
%--------------------------------------
\usepackage{color} %% это для отображения цвета в коде
\usepackage{listings} %% собственно, это и есть пакет listings
\usepackage{caption}

\DeclareCaptionFont{white}{\color{white}} %% это сделает текст заголовка белым
%код ниже нарисует серую рамочку вокруг заголовка кода.
\DeclareCaptionFormat{listing}{\colorbox{gray}{\parbox{\textwidth}{#1#2#3}}}
\captionsetup[lstlisting]{format=listing,labelfont=white,textfont=white}
%--------------------------------------
\definecolor{lemonchiffon}{rgb}{1.0, 0.98, 0.8}
\newtcolorbox{mainblock}{colback=lemonchiffon,grow to right by=-10mm,grow to left by=-10mm,
boxrule=0pt,boxsep=0pt,breakable} % настройки области с изменённым фоном
%--------------------------------------
\definecolor{block-gray}{gray}{0.95} % уровень прозрачности (1 - максимум)
\newtcolorbox{importantblock}{colback=block-gray,grow to right by=-10mm,grow to left by=-10mm,
boxrule=0pt,boxsep=0pt,breakable} % настройки области с изменённым фоном
%--------------------------------------
\makeatletter
\newcommand{\settag}[1]{
  \tag*{(#1)\qquad}
  \edef\@currentlabel{\theequation#1}}
\makeatother

%---------code stuff-------------------
\lstset{
language=C++,                 % выбор языка для подсветки (здесь это С)
basicstyle=\footnotesize\sffamily, % размер и начертание шрифта для подсветки кода
numbers=left,               % где поставить нумерацию строк (слева\справа)
numberstyle=\tiny,           % размер шрифта для номеров строк
stepnumber=1,                   % размер шага между двумя номерами строк
numbersep=5pt,                % как далеко отстоят номера строк от подсвечиваемого кода
backgroundcolor=\color{white}, % цвет фона подсветки - используем \usepackage{color}
showspaces=false,            % показывать или нет пробелы специальными отступами
showstringspaces=false,      % показывать или нет пробелы в строках
showtabs=false,             % показывать или нет табуляцию в строках
frame=single,              % рисовать рамку вокруг кода
tabsize=2,                 % размер табуляции по умолчанию равен 2 пробелам
captionpos=t,              % позиция заголовка вверху [t] или внизу [b]
breaklines=true,           % автоматически переносить строки (да\нет)
breakatwhitespace=false, % переносить строки только если есть пробел
escapeinside={\%*}{*)}   % если нужно добавить комментарии в коде
}
%--------------------------------------

%---------TITLE PAGE---------
\title{Вычислительная математика \\ Лабораторная работа 2}
\author{Ярослав Пылаев, гр. 3530904/80001}
\date{March 31, 2020}

\begin{document}
\maketitle
\newpage

\section{Задание}
\noindent \textbf{Вариант 14.} Семейство матриц зависит от параметра $p$:
\begin{equation}
  \begin{pmatrix}
  %all '--' are changed by \text{--}
    p\text{+}6 & 2 & 6 & 8 & \text{--}2 & 1 & 8 & \text{--}5 \\
    6 & \text{--}22 & \text{--}2 & \text{--}1 & 0 & 5 & \text{--}6 & 4 \\
    \text{--}2 & \text{--}3 & \text{--}16 & 0 & 0 & \text{--}4 & 2 & \text{--}5 \\
    1 & 1 & 4 & 9 & 1 & 0 & 0 & \text{--}6 \\
    0 & 2 & 0 & 2 & \text{--}3 & \text{--}5 & 7 & 5 \\
    6 & \text{--}2 & \text{--}4 & 2 & \text{--}8 & \text{--}12 & 3 & \text{--}3 \\
    \text{--}6 & \text{--}6 & 0 & \text{--}8 & 0 & 5 & \text{--}15 & 0 \\
    0 & 7 & 6 & 0 & \text{--}5 & \text{--}8 & \text{--}5 & \text{--}3
  \end{pmatrix}
\end{equation}
Используя программы $\verb|DECOMP|$ и $\verb|SOLVE|$, вычислить обратные
      матрицы $A^{-1} \hspace{2mm}(p = 1.0, 0.1, 0.01, 0.0001, 0.000001)$ и
      исследовать связь числа обусловленности ($cond$) и нормы матрицы невязки $R = E - AA^{-1}$.

\subsection{Цель работы}
\noindent Для матрицы $(1)$ при заданных значениях параметра $p$ вычилить обратные матрицы
      и исследовать свзяь ее числа обусловленности со знчением нормы матрицы невязки
      $R = E - AA^{-1}$, где $E$ -- едниничная матрица и $A$ -- матрица $(1)$.

\subsection{Задачи}
\begin{enumerate}
  \item Варьировать парметр $p$ исходной матрицы по списку значений $(1.0, 0.1, 0.01,$
        $0.0001, 0.000001)$. На кажом шаге:
  \begin{itemize}
    \item используя программу $\verb|DECOMP|$, получить $LU$-разложение матрицы $(1)$ и
          значение ее числа обусловленности;
    \item используя программу $\verb|SOLVE|$, вычислить обратную матрицу;
    \item получить матрицу невязки и определить значение ее нормы.
  \end{itemize}
  \item По полученным результатам сделать вывод о связи числа обусловленности исходной матрицы
        и нормы матрицы невязки.
\end{enumerate}
\newpage

\section{Теория. Нахождение обратной матрицы с помощью программ DECOMP и SOLVE}
\begin{align*}
  &AX = E, \hspace{5mm} X=A^{-1}, \hspace{3mm} x_k - k\text{-й столбец обратной матрицы}. \\
  &\underbrace{Ax_k = e_k}_{k=1,2,\dots,N}, \hspace{3mm} e_k - \text{$k$-й столбец единичной матрицы}.
\end{align*}
\textit{Однократно} вызывается программа $\verb|DECOMP|$ и строится $LU$-разложение матрицы.
      Далее -- цикл по $k$ от $1$ до $N$:
\begin{align*}
  &e_k \rightarrow B, \\
  &\text{SOLVE}(\dots, A, B, \dots), \\
  &B \rightarrow \text{$k$-й столбец $A^{-1}$}.
\end{align*}

\section{Реализация}
\begin{lstlisting}[label=preparing,caption=Подготовительная часть]
int main()
{
  const int nPValues = 5;
  double arrPValues[nPValues] =
        {1.0, 0.1, 0.01, 0.0001, 0.000001};
  const int mSize = 8 * 8;
  //first value depends by parametr P
  double arrMatrix[mSize] =
      { /*p + */ 6, 2, 6, 8, -2, 1, 8, -5,
        6, -22, -2, -1, 0, 5, -6, 4,
        -2, -3, -16, 0, 0, -4, 2, -5,
        1, 1, 4, 9, 1, 0, 0, -6,
        0, 2, 0, 2, -3, -5, 7, 5,
        6, -2, -4, 2, -8, -12, 3, -3,
        -6, -6, 0, -8, 0, 5, -15, 0,
        0, 7, 6, 0, -5, -8, -5, -3 };
  const int firstElementConstValue = 6;
  //create a new matrix because DECOMP will change the main one
  double arrMatrixCopy[mSize];
  for (int i = 0; i < mSize; ++i) {
    arrMatrixCopy[i] = arrMatrix[i];
  }
  //parametrs for DECOMP and SOLVE
  const int n = sqrt(mSize); //count of rows and cols
  int pivot[n];
  int flag = 0;
  double cond = 0.0;
  //ititialize arrays for matrices
  double arrColumnEMatrix[n];
  double arrInvertMatrix[mSize] = { 0 };
  //the main loop---------
  for (int i = 0; i < nPValues; ++i) {
    arrMatrix[0] = firstElementConstValue + arrPValues[i];
    ......................................................
  }
  //the main loop end---------
\end{lstlisting}
Далее все опреации осуществляются в главном цикле программы (the main loop).

\subsection{DECOMP и SOLVE}
\begin{lstlisting}[label=decompsolve, caption=DECOMP and SOLVE]
  ......................................................
  //---DECOMP---
  decomp(n, n, arrMatrix, &cond, pivot, &flag);
  std::cout << "cond = " << cond << '\n';
  //-----SOLVE
  int tmp = 0;
  for (int j = 0; j < n; ++j) {
    //get current e_k
    for (int k = 0; k < n; ++k) {
      if (k != j) {
        arrColumnEMatrix[k] = 0;
      }
      else {
        arrColumnEMatrix[k] = 1;
      }
    }

    solve(n, n, arrMatrix, arrColumnEMatrix, pivot);

    //fill invert matrix by columns
    for (int k = 0; k < n; ++k) {
      int a = tmp + n * k;
      arrInvertMatrix[a] = arrColumnEMatrix[k];
    }
    ++tmp;
  }
  //SOLVE------
  ......................................................
\end{lstlisting}
\subsubsection{Пример результатов на одном шаге}
\begin{align*}
  \textbf{p = 1.0} \\
  &cond(A) = 579.638, \\
  &A^{-1} = \begin{pmatrix}
    1.0 & 1.094 & 1.463 & \text{--}1.058 & 0.537 & \text{--}2.164 & \text{--}0.734 & 2.529 \\
    1.0 & 1.011 & 1.462 & \text{--}1.001 & 0.638 & \text{--}2.155 & \text{--}0.631 & 2.466 \\
    \text{--}1.0 & \text{--}1.042 & \text{--}1.515 & 0.990 & \text{--}0.669 & 2.153 & 0.615 & \text{--}2.446 \\
    1.0 & 1.073 & 1.504 & \text{--}0.880 & 0.762 & \text{--}2.200 & \text{--}0.609 & 2.487 \\
    \text{--}1.0 & \text{--}0.828 & \text{--}1.247 & 0.845 & \text{--}0.720 & 1.750 & 0.312 & \text{--}1.999 \\
    1.0 & 0.896 & 1.310 & \text{--}0.897 & 0.633 & \text{--}1.919 & \text{--}0.442 & 2.110 \\
    \text{--}1.0 & \text{--}1.116 & \text{--}1.535 & 0.994 & \text{--}0.665 & 2.262 & 0.657 & \text{-2.621} \\
    1.0 & 1.127 & 1.525 & \text{--}1.029 & 0.772 & \text{--}2.293 & \text{--}0.679 & 2.599
  \end{pmatrix}.
\end{align*}
\\
\subsection{Матрица невязки}
\begin{lstlisting}[label=matrixnorm, caption=Residual matrix]
  ......................................................
  //update spoiled matrix after DECOMP & SOLVE
  for (int j = 0; j < mSize; ++j) {
    arrMatrix[j] = arrMatrixCopy[j];
  }
  arrMatrix[0] = firstElementConstValue + arrPValues[i];
  //compute A * A^{-1} = E
  double arrEMatrix[mSize] = { 0 };
  multiplyMatrices(n, n, arrMatrix, n, n, arrInvertMatrix, n, n, arrEMatrix);
  //compute R = E - A * A^{-1}
  int index = 0;
  for (int j = 0; j < n; ++j) {
    for (int k = 0; k < n; ++k) {
      index = j * n + k;
      if (j != k) {
        arrEMatrix[index] = 0 - arrEMatrix[index];
      }
      else {
        arrEMatrix[index] = 1 - arrEMatrix[index];
      }
    }
  }
  computeNorm(n, n, arrEMatrix);
  ......................................................
\end{lstlisting}
\subsubsection{Пример результатов на одном шаге}

  \small
  \[R \footnotemark = \begin{pmatrix}
    \text{--}2 \cdot 10^{-15} & 0 & \text{--}2 \cdot 10^{-15} & 8 \cdot 10^{-16} &
          0 & \text{--}2 \cdot 10^{-15} & 0 & 0 \\
    0 & 0 & \text{--}4 \cdot 10^{-15} & \text{--}3 \cdot 10^{-15} & \text{--}3 \cdot 10^{-15} &
          2 \cdot 10^{-15} & 2 \cdot 10^{-15} & \text{--}1 \cdot 10^{-14} \\
    2 \cdot 10^{-15} & 9 \cdot 10^{-16} & 4 \cdot 10^{-15} & \text{--}9 \cdot 10^{-16} &
          9 \cdot 10^{-16} & \text{--}2 \cdot 10^{-15} & \text{--}9 \cdot 10^{-16} & 0 \\
    3 \cdot 10^{-15} & 2 \cdot 10^{-15} & 4 \cdot 10^{-15} & \text{--}9 \cdot 10^{-16} &
          2 \cdot 10^{-15} & \text{--}7 \cdot 10^{-15} & \text{--}2 \cdot 10^{-15} & 4 \cdot 10^{-15} \\
    \text{--}3 \cdot 10^{-15} & \text{--}3 \cdot 10^{-15} & \text{--}6 \cdot 10^{-15} & 3 \cdot 10^{-15} &
          \text{--}2 \cdot 10^{-15} & 2 \cdot 10^{-15} & 2 \cdot 10^{-15} & \text{--}5 \cdot 10^{-15} \\
    \text{--}1 \cdot 10^{-15} & \text{--}4 \cdot 10^{-15} & \text{--}6 \cdot 10^{-15} & \text{--}3 \cdot 10^{-15} &
          \text{--}2 \cdot 10^{-15} & \text{--}5 \cdot 10^{-15} & 9 \cdot 10^{-16} & \text{--}4 \cdot 10^{-15} \\
    4 \cdot 10^{-15} & 4 \cdot 10^{-15} & 4 \cdot 10^{-15} & 0 & 2 \cdot 10^{-15} & 0 & 4 \cdot 10^{-15} & 0 \\
    \text{--}4 \cdot 10^{-15} & 0 & \text{--}4 \cdot 10^{-15} & 9 \cdot 10^{-16} &
          4 \cdot 10^{-15} & 0 & 9 \cdot 10^{-16} & \text{--}2 \cdot 10^{-15} \\
  \end{pmatrix} \]
\footnotetext{Для удобства анализа значения элементов матрицы невязки округлены до целых.}
\normalsize
Норма матрицы $=$ $2.931 \cdot 10^{-14}$
\subsubsection{Комментарии}
\begin{itemize}
  \item Норма матрицы рассчитывается по формуле \[\parallel A \parallel = max_i \sum_{j=1}^n |a_{ij}|.\]
  \item Для умножения матриц была написана функция $\verb|multiplyMatrices(...)|$, ее код приведен в соответствующем разделе ниже.
\end{itemize}

\subsection{Результаты}
\begin{tabular}{ | c | c | c | c | c | c |}
  \hline
  \footnotesize
  $P$ & 1.0 & 0.1 & 0.01 & 0.0001 & 0.000001 \\ \hline
  $Cond$ & 579.638 & 6013.36 & 60367.3 & $6 \cdot 10^6$ & $6 \cdot 10^8$ \\ \hline
  $Norm$ & $2.9 \cdot 10^{-14}$ & $2.6 \cdot 10^{-13}$ & $4.7 \cdot 10^{-12}$ & $3.5 \cdot 10^{-10}$ &
        $3.5 \cdot 10^{-8}$ \\ \hline
\end{tabular} \\

\section{Вывод}
\noindent Таблица результатов демонстрирует связь между параметром $p$, числом оусловленности исходной матрицы
      и нормой матрицы невязки. Мы можем наблюдать, что, чем меньше значение параметра,
      тем больше число обусловленности и тем больше же и норма. А рост нормы может оказаться
      критическим при требуемой небольшой погрешности вычислений. \\

\noindent Чтобы убедиться, продолжим уменьшать парметр $p$: \\

\begin{tabular}{ | c | c | c | c | c |}
  \hline
  \footnotesize
  $P$ & $10^{-7}$ & $10^{-8}$ & $10^{-9}$ & $10^{-10}$ \\ \hline
  $Cond$ & $6 \cdot 10^9$ & $6 \cdot 10^{10}$ & $6 \cdot 10^{11}$ & $6 \cdot 10^{12}$ \\ \hline
  $Norm$ & $3.5 \cdot 10^{-7}$ & $3.2 \cdot 10^{-6}$ & $4 \cdot 10^{-15}$ & \textbf{0.0003} \\ \hline
\end{tabular} \\

\noindent А ведь мы меняем только первое значение исходной матрицы 8$\times$8.
      Таким образом, при расчетах нужно быть крайне внимательными к изменениям даже одного матричного элемента.

\section{Дополнительные процедуры}
\begin{lstlisting}[label=multiplyMatrices, caption=Multiply Matrices]
  bool multiplyMatrices(const int rows1, const int cols1,
        const double *arrMatrix1, const int rows2,
        const int cols2, const double *arrMatrix2,
        const int rowsRes, const int colsRes,
        double *arrMatrixRes)
  {
    if (rows1 <= 0 || cols1 <= 0 || rows2 <= 0 || cols2 <= 0
          || rowsRes <= 0 || colsRes <= 0) {
      throw std::invalid_argument("...");
      return 1;
    }
    if (cols1 != rows2) {
      throw std::invalid_argument("...");
      return 1;
    }
    if (rowsRes != rows1 || colsRes != cols2) {
      throw std::invalid_argument("...");
      return 1;
    }
    int index = 0;
    for (int i = 0; i < rowsRes; ++i) {
      for (int j = 0; j < colsRes; ++j) {
        index = i * colsRes + j;
        arrMatrixRes[index] = 0;
        for (int k = 0; k < cols1; ++k) {
          arrMatrixRes[index] += arrMatrix1[i * cols1 + k]
                * arrMatrix2[k * cols2 + j];
        }
      }
    }
    return 0;
  }
\end{lstlisting}
\begin{lstlisting}[label=multiplyMatrices, caption=Compute Norm]
  double computeNorm(const int rows, const int cols,
                     double *matrix)
  {
    if (rows <= 0 || cols <= 0) return 0.0;
    double max = 0.0;
    double sum = 0.0;
    for (int i = 0; i < rows; ++i) {
      for (int j = 0; j < cols; ++j)
        sum += fabs(matrix[i * cols + j]);
      if (sum > max) max = sum;
      sum = 0.0;
    }
    return max;
  }
\end{lstlisting}


\end{document}
